%%%%%%%%%%%%%%%%%%%%%%%%%%%%%%%%%%%%%%%%%%%%%%%%%%%%%%
%%%%%%%%%%%%%%%%%%%%%%%%%%%%%%%%%%%%%%%%%%%%%%%%%%%%%%
\section[Conclusions]{Concluding remarks}
\subsection{}
%%%%%%%%%%%%%%%%%%%%%%%%%%%%%%%%%%%%%%%%%%%%%%%%%%%%%%
\begin{frame}{}
\begin{exampleblock}{Conclusions}
\begin{itemize}
\item Gaussian Processes offer a mathematically funded and versatile framework for building statistical models.
\item The main assumptions are: the phenomenon output is Gaussian, functional choice of covariance function (kernel).
\item The statistical model needs physical knowledge: through data + expertise guiding the choice of kernel (which may come from the physical model).
\item The statistical model is in essence complementary to the physical model and typically useful 
for decision making (optimization, uncertainty propagation, \dots).
\end{itemize}
\end{exampleblock}

\end{frame}

