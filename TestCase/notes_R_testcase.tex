\documentclass[12pt]{article}
\usepackage[utf8]{inputenc}
\usepackage{amsmath}
\usepackage{amsfonts}
\usepackage{amssymb}
\usepackage{graphicx}
\usepackage{hyperref}
%\graphicspath{{figures/}}
%\usepackage{color}

%%%%%%%%%%%%%%%%%%%%%%%%%%%%%%%%%%%%%%%%%%%%%%%%%%%%%%%%%%%%%%%%%%%%%
\begin{document}
\title{Metamodels and displacements  inversion \\ R installation notes}
\author{Rodolphe Le Riche, Nicolas Durrande and Valérie Cayol}
%\date{15 february 2017}
\maketitle

\section{TODOs}
\begin{itemize}
\item empty list ;-)
\end{itemize}

\section{Prerequisites}
\begin{enumerate}
\item Have R available on your computer, cf. \url{https://www.r-project.org/}
\item Optionally (but really helpful) have rstudio installed, cf. \url{https://www.rstudio.com}
\item Optional: if you want to load the data that are in matlab format (\texttt{file\_name.mat}), 
install the ``R.matlab'' package (either from Tools / Install Package in rstudio or with the command \texttt{install.packages("ggplot2")}. But you can also load directly the ascii csv file (\texttt{file\_name.csv}) from R
\end{enumerate}

\section{Running the demo step by step}
%\begin{itemize}
%\end{itemize}

\section{Files list}
\begin{itemize}
\item \texttt{mogi\_3D.R}~: calculate displacements on a digital terrain model from a point-wise spherical source.
\item \texttt{plots\_3d\_full\_grid.R}~: Load a csv file (full grid), and plots its 3d data.
\item \texttt{process\_3d\_full\_grid\_from\_matlab.R}~: Load a matlab file (full grid), processes it so that it is plotted and (commented out but working) saved in csv format. Displacements are calculated with \texttt{mogi\_3D.R}.
\item data files ending in \texttt{.mat} (matlab format) or \texttt{.csv} (csv format).
\end{itemize}

\end{document}
